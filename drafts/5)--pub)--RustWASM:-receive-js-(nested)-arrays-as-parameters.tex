\documentclass{article}
\usepackage{listings, listings-rust}
\lstset{
basicstyle=\small\ttfamily,
columns=flexible,
breaklines=true
}

\title{RustWASM: receive js (nested) arrays as parameters.}
\date{8/09/2022}

\begin{document}
\maketitle

\texttt{8 sept 2022}

\section{Introduction}
Due to WASM's memory model, passing arrays (and nested arrays) might reveal to be quite a challenge if you're trying to do things securely.

However, we luckily have the \inl{wasm-bindgen} (and \inl{js-sys}) crates that provide us a friendly interface to work on when exchanging data. I should also mention that whenever you are working with WebAssembly, you should always try to avoid exchanging data between JS and your WASM binary if you are developing with performance as one of your goals.

Wasm functions take and return scalar values as an \inl{ArrayBuffer}. Luckily \inl{wasm-bindgen} helps with a lot of stuff, but it doesn't do everything. Generally, we want WASM to work with heavy computations, query and work with the data in its own linear memory and only return small amounts of data.

Anyways, let's take a look at how you can pass (or return) arrays using the \inl{js_sys::Array} type.

\section{The Cargo.toml}
We have to add \inl{js_sys} to our dependencies (and \inl{wasm-bindgen} if you don't have it already):

\begin{lstlisting}
wasm-bindgen = "0.2.82"
js-sys = "0.3.59"
\end{lstlisting}

\section{Receiving arrays as fn parameters}
I have a \inl{verify_external} function from my \href{https://github.com/heytdep/wasm-groth16-verifier}{wasm-groth16-verifier} crate, the function takes 9 arrays as params, so we can use the \inl{Array} type and infer it as type for our 9 paramaters:

\begin{lstlisting}
  #[wasm_bindgen]
pub fn verify_external(
    vk_a: Array,
    vk_b: Array,
    vk_g: Array,
    vk_d: Array,
    vk_g_abc: Array,
    p_a: Array,
    p_b: Array,
    p_c: Array,
    pub_f: Array,
) -> bool
\end{lstlisting}

Now, I know the lenght of all arrays but one, the \inl{vk_g_abc}, as you might guess, this will be the array causing the most headache, that is because the array will be transformed into a nested array looking like this: \inl{[[a_1, b_1], [a_2, b_2]]} and the lenght of the array can vary and the \inl{Array} type has no implementation for working with nested arrays, every element of the array is the same, regarldess if it's a string or an array. I've decided to manually split the input array which will look like this \inl{[a_1, b_1, a_2, b_2, ...]} into \inl{[[a_1, b_1], [a_2, b_2], ...]}. The algorithm to do so is simple and I've implemented it as follows (it might not be optimized):

\begin{lstlisting}
  let mut abc_vec: Vec<[String; 2]> = Vec::new();
  let mut iters = 0;
  let mut i = 1;
  for _ in 0..vk_g_abc.length() {
    if i != 2 {
      i += 1;
    } else {
      if iters > 0 {
        iters += 1;
        abc_vec.push([
          vk_g_abc.get(iters + (iters - 2)).as_string().unwrap(),
          vk_g_abc.get(iters + (iters - 2) + 1).as_string().unwrap(),
        ]);
        i = 1;
      } else {
        iters += 1;
        abc_vec.push([
          vk_g_abc.get(iters * (i - 2)).as_string().unwrap(),
          vk_g_abc.get(iters * (i - 1)).as_string().unwrap(),
        ]);
        i = 1;
      }
    }
  }
\end{lstlisting}

The loop has to build a new array of two string types and push it to the main \inl{abc_vec} array every two \inl{_}.

You can also see here that we get the elements of the array with \in{arr.get(idx).as_string().unwrap()}. For arrays where I already know the length and indexes of the elements as well, I've simply built the nested tuples like this:

\begin{lstlisting}
  (
  [
    &vk_b.get(0).as_string().unwrap(),
    &vk_b.get(1).as_string().unwrap(),
  ],
  [
    &vk_b.get(2).as_string().unwrap(),
    &vk_b.get(3).as_string().unwrap(),
  ],
  ),
\end{lstlisting}

As you can see, we haven't really received nested arrays, but this shows how you can easily go from a non-nested fixed or non fixed-length array, into a nested one.

You can look at the js demo of passing arrays to WASM of this specific use-case at \url{https://heytdep.github.io/wasm-groth16-verifier}.

\end{document}
