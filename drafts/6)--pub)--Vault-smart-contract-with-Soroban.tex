\documentclass{article}
\usepackage{listings, listings-rust}
\usepackage{graphicx}
\graphicspath{ {../} }
\lstset{
basicstyle=\small\ttfamily,
columns=flexible,
breaklines=true
}

\newcommand{\inl}[1]{\lstinline{#1}}

\title{Built a Vault Smart Contract with Soroban}
\date{8/09/2022}

\begin{document}
\maketitle

\texttt{12 october 2022}

\section{Introduction}
In about August 2022 the Stellar Development Foundation released the first documented version of the sorban_sdk, a Rust SDK to build and test smart contracts compatible with Soroban’s WebAssembly VM. Now that the \inl{v0.1.0} version has been released, I’ve decided to write a detailed write-up about creating a vault smart contract with the soroban SDK.

Without further introductory words, let’s get started!

\section{Contract Workflow}
This contract is of a very popular kind in the DeFi world: a protocol generates yield, and users can buy shares of the protocol's vault, and burn them to get their money back along with the farmed yield (hopefully a positive number).

Since soroban hasn't yet launched on chain, we can't use any yield-farming techniques, on the other, hand we will focus on the vault itself: how to create a contract that allows users to deposit and withdraw shares, all while enforcing composability.

This article will also make use of basic financial maths, luckily this blog post was written in \LaTeX, so we can enjoy equations with a pleasant interface.

As I mentioned, the contract will try to enforce high composability standards, so that it can be used by other contracts very easily, that is why some concepts such as vault admins and not needing deposit confirmation to mint shares to a user may sound strange, but I'll try to explain them here. The contract will have an admin which has to be the invoker of the contract functions. For instance, if Bob wants to deposit to the vault, he calls contract \( contract (args) \) which makes sure Bob has deposited funds into the vault, and then being the admin of the vault calls our vault contract \( vault (args) \).

\section{Setup}
Assuming you have followed the instructions to \href{https://www.rust-lang.org/tools/install}{install Rust}, you can start by firing up your terminal and creating a new crate with cargo:

\begin{lstlisting}
cargo new --lib soroban-vault-contract
cd soroban-vault-contract
\end{lstlisting}


\subsection{The Cargo.toml}
You'll want your \inl{cargo.toml} file to look like this:

\begin{lstlisting}
[package]
name = "soroban-vault-contract"
version = "0.0.0"
edition = "2021"
publish = false

[lib]
crate-type = ["cdylib", "rlib"]

[features]
testutils = ["soroban-sdk/testutils", "soroban-auth/testutils"]

[dependencies]
soroban-sdk = "0.1.0"
soroban-auth = "0.1.0"

[dev_dependencies]
soroban-sdk = { version = "0.1.0", features = ["testutils"] }
soroban-auth = { version = "0.1.0", features = ["testutils"] }
rand = { version = "0.7.3" }

\end{lstlisting}

Here you are telling cargo that you'll be using the soroban sdk and its auth helpers, and you are telling the compiler that the compilation will produce a dynamic system library ("cdylib") and statistically linked executables ("rlib" contents are inserted in the executable) (remember we are building a WASM crate).

\subsection{Project structure}
Besides the \inl{lib.rs} file, in order to make the project a bit cleaner you'll have to create the \inl{test.rs} and \inl{testutils.rs} files:

\begin{lstlisting}
  touch test.rs
  touch testutils.rs
\end{lstlisting}

You're all set now and can start developing the smart contract!

\section{Developing the Smart Contract}

First of all, we have to declare that we aren't going to use Rust's standard library, then import the creates we will be using and declare our test files:

\begin{lstlisting}
#![no_std]

#[cfg(feature = "testutils")]
extern crate std;

mod test;
pub mod testutils;

use soroban_auth::{Identifier, Signature};
use soroban_sdk::{contractimpl, contracttype, BigInt, BytesN, Env};

mod token {
    soroban_sdk::contractimport!(file = "./soroban_token_spec.wasm");
}

\end{lstlisting}

As you might have noticed, we are importing the \inl{./soroban_token_spec.wasm} WebAssembly binary, which is an implementation of the standard token contract. You'll need to copy this binary to our \inl{soroban-vault-contract} directory (you can find the binary in the sorban-examples github repo).

We can now define an enum to use as key for storing data in the contract (rememeber the soroban uses a Key-Value store for its contracts):

\begin{lstlisting}
#[derive(Clone)]
#[contracttype]
pub enum DataKey {
    TokenId,
    Admin,
    TotSupply,
    Balance(Identifier),
    Nonce(Identifier),
}
\end{lstlisting}

This means that we will store 5 kinds of data on our contracts:

\begin{itemize}
\item the \inl{TokenId} will be the id of the token the admin wishes to be used in the vault
\item the \in{Admin} as an identifier for the vault admin
\item the \inl{TotSupply}, intuitively the total supply of vault shares
\item the \inl{Balance} for every \inl{Identifier}, thus the amount of vault shares each user holds
\item the admin's \inl{Nonce}, a cryptographic value that we use in contracts to avoid some actions to be repeated.
\end{itemize}

I have also added a struct to wrap the auth (signature and nonce):

\begin{lstlisting}
#[derive(Clone)]
#[contracttype]
pub struct Auth {
    pub sig: Signature,
    pub nonce: BigInt,
}
\end{lstlisting}

\subsection{Data Helpers}
Below I defined some functions to help me when working with contract data operations (setting and reading data):

\begin{lstlisting}
fn get_contract_id(e: &Env) -> Identifier {
    Identifier::Contract(e.get_current_contract())
}

fn put_tot_supply(e: &Env, supply: BigInt) {
    let key = DataKey::TotSupply;
    e.data().set(key, supply);
}

fn get_tot_supply(e: &Env) -> BigInt {
    let key = DataKey::TotSupply;
    e.data().get(key).unwrap_or(Ok(BigInt::zero(&e))).unwrap()
}

fn put_id_balance(e: &Env, id: Identifier, amount: BigInt) {
    let key = DataKey::Balance(id);
    e.data().set(key, amount);
}

fn get_id_balance(e: &Env, id: Identifier) -> BigInt {
    let key = DataKey::Balance(id);
    e.data().get(key).unwrap_or(Ok(BigInt::zero(&e))).unwrap()
}

fn put_token_id(e: &Env, token_id: BytesN<32>) {
    let key = DataKey::TokenId;
    e.data().set(key, token_id);
}

fn get_token_id(e: &Env) -> BytesN<32> {
    let key = DataKey::TokenId;
    e.data().get(key).unwrap().unwrap()
}
\end{lstlisting}

As you can see, what I do here is just summarize actions like getting the total supply in a single function so that the development results faster and more comfortably.

\subsection{Token Functions}
For our contract, we will only need the standard token contract to check the vault balance, and to transfer funds out of the vault. I have defined two functions that achieve this behavior:

\begin{lstlisting}
fn get_token_balance(e: &Env) -> BigInt {
    let contract_id = get_token_id(e);
    token::Client::new(e, contract_id).balance(&get_contract_id(e))
}

fn transfer(e: &Env, to: Identifier, amount: BigInt) {
    let client = token::Client::new(e, get_token_id(e));
    client.xfer(
        &Signature::Invoker,
        &client.nonce(&Signature::Invoker.identifier(e)),
        &to,
        &amount,
    );
}
\end{lstlisting}

Remember that we are getting the token to transfer and to check the balance from thanks to the \inl{get_token_id()} function we defined previously.


\subsection{Admin management}
Below are the data functions that take care of managing the admin for the vault, these are almost standard in most of soroban examples as of now:

\begin{lstlisting}
fn has_administrator(e: &Env) -> bool {
    let key = DataKey::Admin;
    e.data().has(key)
}

fn read_administrator(e: &Env) -> Identifier {
    let key = DataKey::Admin;
    e.data().get_unchecked(key).unwrap()
}

fn write_administrator(e: &Env, id: Identifier) {
    let key = DataKey::Admin;
    e.data().set(key, id);
}

pub fn check_admin(e: &Env, auth: &Signature) {
    let auth_id = auth.identifier(e);
    if auth_id != read_administrator(e) {
        panic!("not authorized by admin")
    }
}
\end{lstlisting}

\subsection{Nonce management}
Below are the data functions that take care of reading and updating the nonce once read, these are also standard for most soroban examples.

\begin{lstlisting}
fn read_nonce(e: &Env, id: &Identifier) -> BigInt {
    let key = DataKey::Nonce(id.clone());
    e.data()
        .get(key)
        .unwrap_or_else(|| Ok(BigInt::zero(e)))
        .unwrap()
}

fn verify_and_consume_nonce(e: &Env, auth: &Signature, expected_nonce: &BigInt) {
    match auth {
        Signature::Invoker => {
            if BigInt::zero(e) != expected_nonce {
                panic!("nonce should be zero for Invoker")
            }
            return;
        }
        _ => {}
    }

    let id = auth.identifier(e);
    let key = DataKey::Nonce(id.clone());
    let nonce = read_nonce(e, &id);

    if nonce != expected_nonce {
        panic!("incorrect nonce")
    }
    e.data().set(key, &nonce + 1);
}
\end{lstlisting}

\subsection{The Actual Contract Invocation functions}
We can now finally start writing our invocation functions, i.e the functions that other contracts will invoke. For better code maintainability, we define a trait for our vault contract (the comments on top of each function describe what the function does):

\begin{lstlisting}
pub trait VaultContractTrait {
    // Sets the admin and the vault's token id
    fn initialize(e: Env, admin: Identifier, token_id: BytesN<32>);

    // Returns the nonce for the admin
    fn nonce(e: Env) -> BigInt;

    // deposit shares into the vault: mints the vault shares to "from"
    fn deposit(e: Env, auth: Auth, from: Identifier, amount: BigInt);

    // withdraw an ammount of the vault's token id to "to" by burning shares
    fn withdraw(e: Env, auth: Auth, to: Identifier, shares: BigInt);

    // get vault shares for a user
    fn get_shares(e: Env, id: Identifier) -> BigInt;
}
\end{lstlisting}

Then we create a \inl{VaultContract} struct, and write the contract implementation with \inl{#[contractimpl]}:

\begin{lstlisting}

pub struct VaultContract;

#[contractimpl]
impl VaultContractTrait for VaultContract {
    fn initialize(e: Env, admin: Identifier, token_id: BytesN<32>) {
        if has_administrator(&e) {
            panic!("admin is already set");
        }

        write_administrator(&e, admin);

        put_token_id(&e, token_id)
    }

    fn nonce(e: Env) -> BigInt {
        read_nonce(&e, &read_administrator(&e))
    }

    fn deposit(e: Env, admin_auth: Auth, from: Identifier, amount: BigInt) {
        check_admin(&e, &admin_auth.sig);
        verify_and_consume_nonce(&e, &admin_auth.sig, &admin_auth.nonce);

        let tot_supply = get_tot_supply(&e);

        let shares = if BigInt::zero(&e) == tot_supply {
            amount
        } else {
            (amount.clone() * tot_supply) / (get_token_balance(&e) - amount)
        };

        mint_shares(&e, from, shares);
    }

    fn get_shares(e: Env, id: Identifier) -> BigInt {
        e.data()
            .get(DataKey::Balance(id))
            .unwrap_or(Ok(BigInt::zero(&e)))
            .unwrap()
    }

    fn withdraw(e: Env, admin_auth: Auth, to: Identifier, shares: BigInt) {
        check_admin(&e, &admin_auth.sig);
        verify_and_consume_nonce(&e, &admin_auth.sig, &admin_auth.nonce);

        let tot_supply = get_tot_supply(&e);
        let amount = (shares.clone() * get_token_balance(&e)) / tot_supply;

        burn_shares(&e, to.clone(), shares);
        transfer(&e, to, amount);
    }
}

\end{lstlisting}

That was a big step, so let's break the contract down into 4 parts:

\subsubsection{initialization}
In the \inl{initialize} function, we are checking whether the vault contract already has an administrator, if not we are adding one along with the token to use in the vault (the token id is an array of 32 bytes).

\subsubsection{deposit: shares minting}
The deposit first checks if the invoker is actually the contract admin through the \inl{check_admin} function (and it updates the noce). Then it calculates the amount of shares to mint, which respects the following equation:

\( ( S + s ) \div S = ( a + B ) \div B \)

where S is the total supply, s is the amount of shares to mint, a is the amount of tokens the user has deposited into the vault, and B is the token balance of the vault (without taking into account the tokens deposited by the user).

We can simplify this simple equation like this:

\[ BS + Bs = aS + BS \]
\

\[ Bs = aS \]
\

\[ s = aS \div B \]

Thus the shares to mint are \inl{( amount * tot_supply ) / vault token balance - amount // since we need to subtract the amount the user has already deposited in the vault}, as the contract describes.

Lastly, we are calling a \inl{mint_shares} function that we can define under our helper functions section:

\begin{lstlisting}
fn mint_shares(e: &Env, to: Identifier, shares: BigInt) {
    let tot_supply = get_tot_supply(e);
    let id_balance = get_id_balance(e, to.clone());

    put_tot_supply(e, tot_supply + shares.clone());
    put_id_balance(e, to, id_balance + shares);
}
\end{lstlisting}

Here we are simply updating the user's vault share balance and the total supply.

\subsubsection{withdraw}
Withdrawing follows a similar workflow as depositing: first checks the admin is invoking the contract (and updating the nonce), then calculating the amount of tokens to send to the user with the following equation:

\[ ( S - s ) \div S = ( B - a ) \div B \]

Which we simplify into:

\[ BS - Bs = BS - aS \]
\

\[ - Bs = - aS \]
\

\[ Bs = aS \]
\

\[ a = Bs \div S \]

Thus \inl{(shares * vault token balance) / tot_supply}. Lastly, we are calling a \inl{burn_shares} function, which we can define right below the \inl{mint_shares} function:

\begin{lstlisting}
fn burn_shares(e: &Env, to: Identifier, shares: BigInt) {
    let tot_supply = get_tot_supply(e);
    let id_balance = get_id_balance(e, to.clone());

    assert!(shares < id_balance); // asserting that user is not trying to burn more than he has

    put_tot_supply(e, tot_supply - shares.clone());
    put_id_balance(e, to, id_balance - shares);
}

\end{lstlisting}

\subsubsection{getters}
The nonce and \inl{get_balance} fns are simple getters that simply read data from the contract and extern it to the invocation.

\section{Testing the Contract}
In this section, we are going to test our contract and simulate the situation where:
\begin{enumerate}
\item User 1 is the admin of the vault
\item User 1 buys 5 shares from the vault
\item User 2 buys 8 shares from the vault
\item the vault earns a yield of 100\% (simply minting 13 tokens to the contract)
\item user 1 withdraws 3 of 5 shares from the vault
\end{enumerate}

\subsection{testutils.rs}
In order to have a better testing experience, we are going to use some testutils to ease the process of registering the contract and invoking it.

To interact with the Vault contract, we will have to import the \inl{VaultContractClient} struct, which is added automatically to our crate:

\begin{lstlisting}
#![cfg(any(test, feature = "testutils"))]

use crate::{Auth, VaultContractClient};
use soroban_auth::{Identifier, Signature};

use soroban_sdk::{AccountId, BigInt, BytesN, Env};

pub fn register_test_contract(e: &Env, contract_id: &[u8; 32]) {
    let contract_id = BytesN::from_array(e, contract_id);
    e.register_contract(&contract_id, crate::VaultContract {});
}

\end{lstlisting}

Here I have also defined a function to register the contract in the environment.

Now we have to define our testutils \inl{VaultContract} struct, write a \inl{client()} implementation which will return the client we imported from the crate root, then write a \inl{new()} implementation which simply builds us the contract struct, and finally write all the implementations that invoke the contract functions in the \inl{VaultContractTrait} through the \inl{VaultContractClient}:

\begin{lstlisting}

pub struct VaultContract {
    env: Env,
    contract_id: BytesN<32>,
}

impl VaultContract {
    fn client(&self) -> VaultContractClient {
        VaultContractClient::new(&self.env, &self.contract_id)
    }

    pub fn new(env: &Env, contract_id: &[u8; 32]) -> Self {
        Self {
            env: env.clone(),
            contract_id: BytesN::from_array(env, contract_id),
        }
    }

    pub fn initialize(&self, admin: &Identifier, token_id: &[u8; 32]) {
        self.client()
            .initialize(admin, &BytesN::from_array(&self.env, token_id));
    }

    pub fn nonce(&self) -> BigInt {
        self.client().nonce()
    }

    pub fn deposit(&self, admin: AccountId, from: Identifier, amount: BigInt) {
        self.env.set_source_account(&admin);
        self.client().deposit(
            &Auth {
                sig: Signature::Invoker,
                nonce: BigInt::zero(&self.env),
            },
            &from,
            &amount,
        )
    }

    pub fn withdraw(&self, admin: AccountId, to: Identifier, shares: BigInt) {
        self.env.set_source_account(&admin);
        self.client().withdraw(
            &Auth {
                sig: Signature::Invoker,
                nonce: BigInt::zero(&self.env),
            },
            &to,
            &shares,
        )
    }

    pub fn get_shares(&self, id: &Identifier) -> BigInt {
        self.client().get_shares(id)
    }
}

\end{lstlisting}

Note that I am using \inl{Signature::Invoker} as signature to pass, this means that I am passing the signature of the source account of the transaction, which I set to be the admin with \inl{self.env.set_source_account(&admin)}.

We are now ready to write the actual test.

\subsection{test.rs}
The first thing we want to do in our test.rs file, is importing the testutils, the standard token implementation we imported in the lib.rs file, and then the sorboan sdk with its auth helpers (along with the rand crate to generate random numbers for our contract ids). Then we also define three functions:

\begin{itemize}
\item one to generate random contract ids
\item one to register the token contract of USDC (just a representation of it)
\item one to create the vault contract through the testutils' \inl{VaultContract} struct
\end{itemize}

\begin{lstlisting}
#![cfg(test)]

use crate::testutils::{register_test_contract as register_vault, VaultContract};
use crate::token::{self, TokenMetadata};
use rand::{thread_rng, RngCore};
use soroban_auth::{Identifier, Signature};
use soroban_sdk::{testutils::Accounts, AccountId, BigInt, BytesN, Env, IntoVal};

fn generate_contract_id() -> [u8; 32] {
    let mut id: [u8; 32] = Default::default();
    thread_rng().fill_bytes(&mut id);
    id
}

fn create_token_contract(e: &Env, admin: &AccountId) -> ([u8; 32], token::Client) {
    let id = e.register_contract_token(None);
    let token = token::Client::new(e, &id);
    // decimals, name, symbol don't matter in tests
    token.init(
        &Identifier::Account(admin.clone()),
        &TokenMetadata {
            name: "USD coin".into_val(e),
            symbol: "USDC".into_val(e),
            decimals: 7,
        },
    );
    (id.into(), token)
}

fn create_vault_contract(
    e: &Env,
    admin: &AccountId,
    token_id: &[u8; 32],
) -> ([u8; 32], VaultContract) {
    let id = generate_contract_id();
    register_vault(&e, &id);
    let vault = VaultContract::new(e, &id);
    vault.initialize(&Identifier::Account(admin.clone()), token_id);
    (id, vault)
}
\end{lstlisting}

Finally, let's emulate the situation I've described at the beginning of this section. The code below is quite intuitive and each action is commented, so I won't paraphrase it. I only would like to shine light on how I am invoking the token contract: we are using the \inl{with_source_account()} method and then \inl{Signature::Invoker} as signature. This means that we are signing the transaction with the source account we set.

\begin{lstlisting}
  
#[test]
fn test() {
    let e: Env = Default::default();
    let admin1 = e.accounts().generate(); // generating the usdc admin

    let user1 = e.accounts().generate();
    let user2 = e.accounts().generate();
    let user1_id = Identifier::Account(user1.clone());
    let user2_id = Identifier::Account(user2.clone());

    let (contract1, usdc_token) = create_token_contract(&e, &admin1); // registered and initialized the usdc token contract
    let (contract_vault, vault) = create_vault_contract(&e, &user1, &contract1); // registered and initialized the vault token contract, with usdc as vault token

    let vault_id = Identifier::Contract(BytesN::from_array(&e, &contract_vault)); // the id of the vault

    // minting 1000 usdc to user1
    usdc_token.with_source_account(&admin1).mint(
        &Signature::Invoker,
        &BigInt::zero(&e),
        &user1_id,
        &BigInt::from_u32(&e, 1000),
    );

    // minting 1000 usdc to user2
    usdc_token.with_source_account(&admin1).mint(
        &Signature::Invoker,
        &BigInt::zero(&e),
        &user2_id,
        &BigInt::from_u32(&e, 1000),
    );

    // user 1 deposits 5 usdc into vault
    usdc_token.with_source_account(&user1).xfer(
        &Signature::Invoker,
        &BigInt::zero(&e),
        &vault_id,
        &BigInt::from_u32(&e, 5),
    );

    // user1 buys shares from the vault
    vault.deposit(user1.clone(), user1_id.clone(), BigInt::from_i32(&e, 5));
    assert_eq!(
        usdc_token.with_source_account(&admin1).balance(&user1_id),
        995
    );
    assert_eq!(vault.get_shares(&user1_id), 5);

    // user 2 deposits 8 usdc into vault
    usdc_token.with_source_account(&user2).xfer(
        &Signature::Invoker,
        &BigInt::zero(&e),
        &vault_id,
        &BigInt::from_u32(&e, 8),
    );
    // user2 buys shares from the vault
    vault.deposit(user1.clone(), user2_id, BigInt::from_i32(&e, 8));

    // the vault generates yield
    usdc_token.with_source_account(&admin1).mint(
        &Signature::Invoker,
        &BigInt::zero(&e),
        &vault_id,
        &BigInt::from_u32(&e, 13),
    );

    // user1 withdraws from the vault
    vault.withdraw(user1, user1_id.clone(), BigInt::from_i32(&e, 3));
    assert_eq!(
        usdc_token.with_source_account(&admin1).balance(&user1_id),
        1001
    ); // user 1 now has 1001 USDC and still has 2 shares in the vault.
    assert_eq!(vault.get_shares(&user1_id), 2);
}

\end{lstlisting}

\section{Wrapping it all up}
I hope that the reader found this tutorial useful, and in order to make the process easier if you encounter errors you shouldn't be encountering I have published the code to a \href{https://github.com/Xycloo/soroban-vault-contract}{Github repo}.

You can go ahead and try it out:

\begin{lstlisting}
> git clone https://github.com/Xycloo/soroban-vault-token-contract
> cd soroban-vault-contract

> cargo test

     Finished test [unoptimized + debuginfo] target(s) in 0.03s
     Running unittests src/lib.rs (target/debug/deps/soroban_vault_contract-9f7cb3a1d823b9eb)

running 1 test
test test::test ... ok

test result: ok. 1 passed; 0 failed; 0 ignored; 0 measured; 0 filtered out; finished in 0.01s

   Doc-tests soroban-vault-contract

running 0 tests

test result: ok. 0 passed; 0 failed; 0 ignored; 0 measured; 0 filtered out; finished in 0.00s

  
\end{lstlisting}



The test passed, mission accomplished!

\end{document}
