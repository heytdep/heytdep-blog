\documentclass{article}
\usepackage{listings, listings-rust}
\usepackage{graphicx}
\graphicspath{ {../} }
\lstset{
basicstyle=\small\ttfamily,
columns=flexible,
breaklines=true
}

\newcommand{\inl}[1]{\lstinline{#1}}

\title{Building a Lottery Smart Contract with Soroban}
\date{8/09/2022}

\begin{document}
\maketitle

\texttt{16 october 2022}

\section{Introduction}
In this small post, I'll share the code of a Lottery Smart Contract I built with Soroban: Stellar's smart contract platform. This is the second time I write about Soroban, check the first one \href{https://heytdep.github.io/comp_posts/6)--pub)--Vault-smart-contract-with-Soroban/post.html}{here}, where I talk about creating a Vault smart contract while going through every step I make in the development.

On the other hand, in this post I'll just extern how the contract works. By following the other article and by fiddling a bit with the source code you should be able to explore the contract thorough.

\section{Contract Workflow}
A lottery is a perfect example of showcasing the potential of smart contracts: transparency in the code and in its execution + ease of transferring assets. To create a lottery smart contract, we need it to do three things:

\begin{itemize}
\item Set up the lottery \( \rightarrow \) the admin, which coin/token to accept as payment, how many winners there will be, how much will a lottery ticket cost (in terms of the accepted coin/token)
\item Buying the ticket \( \rightarrow \) swapping the required amount of the accepted token for a lottery ticket, thus a possiblity of winning.
\item Running the lottery \( \rightarrow \) When the admin wants (or when a certain condition is triggered), the lottery contract will randomly select and reward as many winners as the admin defined when setting up the lottery.
\end{itemize}

This is a summary of what we are going to do below, rememeber that reading the quickstart examples in the \href{https://soroban.stellar.org/docs/}{Soroban docs} will make understanding this article easier.

\section{Fastrand}

While for the vault contract I haven't used other crates besides the soroban sdk and its auth helpers, you'll notice that the \inl{Cargo.toml} file for this contract uses \inl{fastrand}, which the contract uses to generate random numbers for our winners. This is a good choice since we don't need cryptographically secure numbers and this crate is very light and has 0 dependencies.

\begin{lstlisting}
[dependencies]
soroban-sdk = "0.1.0"
soroban-auth = "0.1.0"
fastrand = "1.8.0"
\end{lstlisting}

\section{Testing the contract}
\textbf{\href{https://github.com/Xycloo/soroban-lottery-contract}{This is the repo}} of the contract.

\begin{lstlisting}
> git clone https://github.com/Xycloo/soroban-lottery-contract
> cd soroban-lottery-contract
> cargo test
   Compiling soroban-vault-contract v0.0.0 (/home/tommasodeponti/Desktop/soroban-lottery-contract)
    Finished test [unoptimized + debuginfo] target(s) in 2.19s
     Running unittests src/lib.rs (target/debug/deps/soroban_vault_contract-3af336ea264d8fce)

running 2 tests
test test::test ... ok


BALANCES
user1: 996
user2: 995
user3: 995 
user4: 995 
user5: 1017
user6: 1017 
user7: 995
user7: 995
user9: 995

test test::test_sequence ... ok

test result: ok. 2 passed; 0 failed; 0 ignored; 0 measured; 0 filtered out; finished in 0.03s

   Doc-tests soroban-vault-contract

running 0 tests

test result: ok. 0 passed; 0 failed; 0 ignored; 0 measured; 0 filtered out; finished in 0.00s

\end{lstlisting}

\

\

contact: \href{mailto:tommasodeponti@zohomail.eu}{tommasodeponti@zohomail.eu}
\end{document}
