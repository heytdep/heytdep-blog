\documentclass{article}
\usepackage{listings, listings-rust}
\lstset{
basicstyle=\small\ttfamily,
columns=flexible,
breaklines=true
}

\title{Groth16 proof verification from scratch in Rust}
\date{28/08/2022}

\begin{document}
\maketitle

\texttt{28 aug 2022}

\section{Introduction}
Groth16 is widely used in zk-SNARKs due to its efficiency in terms of both proving and verifying, and the succinctness of the proof. I've been experimenting on the verification algorithm in order to implement it on a \texttt{#[no_std]} WASM environment.
The implementation is not complete and I've decided to work on a forked version of the \texttt{ark-groth16} library, which I am optimizing and adapting to compile on Stellar's native smart contracts platform: Soroban.
However, I'd like to share here my current progress for both future reference and for whoever wanted to complete the work.

\section{Before verifying}

Before we verify a proof, we need two things:

\begin{enumerate}
\item the proving (needed for proving) and verification key
\item the proof
\end{enumerate}

We obtain the proving and verification key through a trusted setup, a detailed explanation of the setup can be found \href{https://xn--2-umb.com/22/groth16/index.html#trusted-setup}{here}.
The proof is generated starting from a witness, or private field (and) public statements, or public fields, and the proving key. These steps rely on randomly-generated values \( \tau,\alpha,\beta,\gamma \) \ that must not be exposed for the security of the proving/verfying system, since they could allow a potentially malicious attacker to compute a valid proof \( \pi \) with an invalid private field.
Also, all of the computations rely on using three elliptic curve groups and correspondig generators \( G_1, G_2, G_3 \) and a pairing (\url{https://xn--2-umb.com/22/pairings/index.html}). 

I've already built my verification key and my proof, and I want to tell the verifier that I know the two hash images \texttt{i} and \texttt{j} for a hash \texttt{k}, all without the three values to the verifier.

\begin{lstlisting}
  Verification Key:
  
  "alpha": [
    "0x1373fe23372b4855c4f0d6fdbbd5126df7a9cd4461c856ab4f0c4247465e5408",
    "0x22430ada951f2f06c72a12fb8f607655768ec3f403641a112484275583d249ef"
  ],
  "beta": [
    [
      "0x05e9470fd1b46f5bff475d55ed075706c71497a630240d2b71885b560925e111",
      "0x16ac9c10f9097ad05e34504c9a6471c66ce4f4d3a7f6eaf1540ad09bb9e8b08f"
    ],
    [
      "0x23aaf52b99438f46380b12efd78ab573a5aa99484d2464a538da5bfcc2642f8d",
      "0x2451d249a03be22ff043978c64fff1c1ac6cddd05ad749f187487de73e3fe749"
    ]
  ],
  "gamma": [
    [
      "0x20b882fa627e06cc76a663a216c1589bbdee0c0ae9828cfd8fb753bd3805d472",
      "0x2b2a9cea8ea5ce73b6719b883e199d55a97f1ea9328a03011ea68ae20be43d17"
    ],
    [
      "0x303cd5e6886860d4196398aca5c3de1f82f5f2e759859c0ddff2241291a54422",
      "0x083183395558b36834fa86432530f1116596aaac10b69f0b2a1721f3c0b59030"
    ]
  ],
  "delta": [
    [
      "0x249d82794f83ee45108cfbcf5a43312905142a5fc5b9550854168e62a63e2849",
      "0x2e052b450ee870289ef52c9b7e4fcdb0529fdecc675edd397a27907d34928b60"
    ],
    [
      "0x09fccaacd1f52a25b3e49fee4bc9d737cc0006400e93505c317a1cec2f64b346",
      "0x2ffd0e0392b89070eb4c62d6375c971606d43abe5dcfc2672e426a6421e5b1b3"
    ]
  ],
  "gamma_abc": [
    [
      "0x1aa7acafe10d3a96ad803bf2baa0e7c239593dab41ab924242a6a6b44035e9ad",
      "0x2312c4caac75bbc30aa69f07aa36605959250a59e566cf2b27f82bf97a6d1703"
    ]
  ]

  Proof:
  "a": [
      "0x0702f9d104aa1ea7b599b27af9839cc32a74fb5a8a2e2181db9f09829efb35b9",
      "0x21d51abc11310dda486c1c9764e6746cd5facd33eb9c9b156f27cb76cbd9d0ce"
    ],
    "b": [
      [
        "0x07fc10dc59836597b44bcfb1dcbc307fac86c79cbc64f8013beadc2edd5b02a9",
        "0x0a466fd8cc51ede866f0dc5a797ba46759fa5f2c7fb32acb41e6bb46309d1c6c"
      ],
      [
        "0x0b5a03c9e0c6133e9a3ce2fb72bcfa147cb680476aabf8e9748d02233a192a55",
        "0x141555fb6598127a15d2be6f0489227a2989701740c23251c89b8a2133251e46"
      ]
    ],
    "c": [
      "0x0b31d008bda03121c8d5b4fa9ec4e1f0a8086acd6d3e9aa6ce8e7c5621c4462f",
      "0x188340086b171cf34d18dd96c00767531269b8154480daf0d9c96e505c4cbf4b"
  ]
  
\end{lstlisting}

\section{Verifying}

It's now time to verify the proof, if you remember, we only need the triplet \texttt{A}, \texttt{B}, \texttt{C} (proof) and the verification key. We might also have some public inputs, but all of our inputs (\texttt{i}, \texttt{j}, \texttt{k}) are private.

Since we need to store our verification key and proof, we need 4 structs:

\begin{lstlisting}
  struct G1Point(uint256, uint256);
  struct G2Point([uint256; 2], [uint256; 2]);

  struct VerifyingKey {
    alpha: G1Point,
    beta: G2Point,
    gamma: G2Point,
    delta: G2Point,
    gamma_abc: Vec<G1Point>,
  }

  struct Proof {
    a: G1Point,
    b: G2Point,
    c: G1Point,
  }
\end{lstlisting}

The above assumes you have a \texttt{uint256} type, which can be for example a \texttt{BigInt};

\subsection{Preparing}

If you paid attention to my verification key and proof, the two \texttt{VerifyingKey} and \texttt{Proof} structs represent my input. We can now easily construct our verifying key.

\begin{lstlisting}
  impl VerifyingKey {
    fn new() -> Self {
      VerifyingKey {
        alpha: G1Point(
        uint256("0x1373fe23372b4855c4f0d6fdbbd5126df7a9cd4461c856ab4f0c4247465e5408"),
        uint256("0x22430ada951f2f06c72a12fb8f607655768ec3f403641a112484275583d249ef"),
        ),
        beta: G2Point(
        [
          uint256("0x05e9470fd1b46f5bff475d55ed075706c71497a630240d2b71885b560925e111"),
          uint256("0x16ac9c10f9097ad05e34504c9a6471c66ce4f4d3a7f6eaf1540ad09bb9e8b08f"),
        ],
        [
          uint256("0x23aaf52b99438f46380b12efd78ab573a5aa99484d2464a538da5bfcc2642f8d"),
          uint256("0x2451d249a03be22ff043978c64fff1c1ac6cddd05ad749f187487de73e3fe749"),
        ],
        ),
        gamma: G2Point(
        [
          uint256("0x20b882fa627e06cc76a663a216c1589bbdee0c0ae9828cfd8fb753bd3805d472"),
          uint256("0x2b2a9cea8ea5ce73b6719b883e199d55a97f1ea9328a03011ea68ae20be43d17"),
        ],
        [
          uint256("0x303cd5e6886860d4196398aca5c3de1f82f5f2e759859c0ddff2241291a54422"),
          uint256("0x083183395558b36834fa86432530f1116596aaac10b69f0b2a1721f3c0b59030"),
        ],
        ),

        delta: G2Point(
        [
          uint256("0x249d82794f83ee45108cfbcf5a43312905142a5fc5b9550854168e62a63e2849"),
          uint256("0x2e052b450ee870289ef52c9b7e4fcdb0529fdecc675edd397a27907d34928b60"),
        ],
        [
          uint256("0x09fccaacd1f52a25b3e49fee4bc9d737cc0006400e93505c317a1cec2f64b346"),
          uint256("0x2ffd0e0392b89070eb4c62d6375c971606d43abe5dcfc2672e426a6421e5b1b3"),
        ],
        ),
        gamma_abc: vec![
          G1Point(
          uint256("0x1aa7acafe10d3a96ad803bf2baa0e7c239593dab41ab924242a6a6b44035e9ad"),
          uint256("0x2312c4caac75bbc30aa69f07aa36605959250a59e566cf2b27f82bf97a6d1703"),
          )
        ],
    }
  }
  }

  ...
  
  let vk = verifying_key();
  assert_eq!(input.clone().length + 1, vk.gamma_abc.length); // notice how gamma_abc is a vector of one element, and that we have 0 public inputs
\end{lstlisting}

\textit{Note that in these examples I haven't used the Rust standard library, thus the \texttt{Vec} type gets its elements by index through the \texttt{get(idx)} method).}

We should then aggregate the public inputs with something like:

\begin{lstlisting}
  let mut vk_x = G1Point ( 0, 0 );
  let snark_scalar_field =
  uint256("21888242871839275222246405745257275088548364400416034343698204186575808495617");
  
  for i in 0..input.length {  
    assert!(input.get(i).unwrap().unwrap() < snark_scalar_field);
    vk_x = elliptic_curve_addition(
      vk_x,
      elliptic_curve_scalar_mul(
        vk.gamma_abc.get(i + 1).unwrap().unwrap(),
        input.get(i).unwrap().unwrap(),
      ),
    );
  }
\end{lstlisting}

But since we have 0 public inputs, we can skip that part, and simply perform this:

\begin{lstlisting}
  vk_x = elliptic_curve_addition(vk_x, vk.gamma_abc.get(0).unwrap().unwrap())
\end{lstlisting}

My implementations of the elliptic operations used are the following (I don't have a strong math background so they could hold incorrect assumptions, you should double-check them):

\begin{lstlisting}
  fn elliptic_double(pt: G1Point) -> G1Point {
    let x = pt.0;
    let y = pt.1;

    let l = 3 * x.exp(2) / (2 * y);

    let newx = l.exp(2) - 2 * x;
    let newy = -l * newx + l * x - y;

    G1Point(newx, newy)
  }

  fn elliptic_curve_addition(p1: G1Point, p2: G1Point) -> G1Point {
    let x1 = p1.0;
    let y1 = p1.1;
    let x2 = p2.0;
    let y2 = p2.1;

    if x2 == x1 && y2 == y1 {
      return elliptic_double(p1); // if the two points are the same, simply double them
    }
    // case when x2 == x1 => None

    let l = (y2 - y1) / (x2 - x1);

    let newx = l.exp(2) - x1 - x2;
    let newy = -l * newx + l * x1 - y1;

    G1Point(newx, newy)
  }

  fn elliptic_curve_scalar_mul(pt: G1Point, n: u64) -> G1Point {
    if n == 0 {
      return 0;
    } else if n == 1 {
      return pt;
    } else if !(n % 2) {
    return elliptic_curve_scalar_mul(elliptic_double(pt), n / 2);
  } else {
    return elliptic_curve_addition(elliptic_curve_scalar_mul(elliptic_double(pt), n / 2), pt);
  }
  }
\end{lstlisting}

\subsection{Pairing}

EC pairings are how we actually implement verification. We define a pairing function:

\[ e: G_1 \times  G_2 \rightarrow G_t \]

Which isn't degenerate and that respects bilinearity:

\[ e ( m \times P1, n \times P2) = e ( P1, P2 ) ^ ( m * n ) \]

We can then verify that:

\[ e ( a_1, b_1) \times e ( a_2, b_2 ) \times e ( a_3, b_3 ) ... \times e ( a_k, b_k ) = 1 \]

This is how the EIP-197 implementation works.

We aggregate our inputs into (\(L\)), we have to check (starting from the \href{https://xn--2-umb.com/22/pairings/index.html}{pairing}):

\[ e(A, B) = e(\alpha \ \times \ G_1, \ \beta \ \times G_2) + e(L, \ \gamma \ \times \ G_2) + e(C, \ \delta \ \times \ G_2)\]

Where \texttt{A}, \texttt{B}, \texttt{C} are the proof, \texttt{L} is the aggregation of the inputs.

The implementation for the bn128 pairing (by \href{https://vitalik.ca}{vitalik}) can be found \href{https://github.com/ethereum/py_pairing/blob/master/py_ecc/bn128/bn128_pairing.py#L68}{here}.

\

The equation becomes:

\[ P_a \times P_b + ( L_i \div VK_{\gamma} ) \times ( - \gamma ) + P_c \times ( - \delta ) = VK_{\alpha} \times VK_{\beta} \]

Where \( G_1 \times G_2 \) corresponds to \( e ( G_1 , G_2 ) \).

\end{document}
