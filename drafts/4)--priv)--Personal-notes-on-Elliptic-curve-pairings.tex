\documentclass{article}

\title{Personal notes on ECP and ECC}
\newcommand{\inl}[1]{\lstinline{#1}}

\begin{document}
\maketitle

\inl{last updated: 6 Sept 2022}

\section{Pairing basics}

Pairings' power lies in the fact that they allow for checking fairly complicated equations on elliptic curve points (x, y).

\[e (P, Q) * e (G, G * 5) = 1 == p * q + 5 = 0 \]

Pairings in general aren't complicate, for example when working with integers:

\[ e (x, y) = 2^{ x \times y } \]

However, such numbers aren't suitable for ECC since they are very simple to analyze, for example, if you have both \( x \) and \( e (x, y) \), you can compute \( y \) with a division and a logarithm.

Elliptic curve pairings on the other hand, allow you to treat these pairings as black boxes where only certain operations are permitted (+, -, /, *).

\

Prime fields are a set of numbers

\[ { 0, 1, 2, ..., p - 1 } where \ p \ is \ a \ prime \]

that are needed in ECC TK

Operations involving \( p \) are as follows:

\[ a + b:  (a + b) \mod p \]
\[ a - b:  (a - b) \mod p \]
\[ a \times b:  (a \times b) \mod p \]
\[ a \div b:  (a \times b^{p-2}) \mod p \]

The division is different since we only want to deal with integers.

We also work with extensions of a field, where in our case \( i ^ 2 + 1 = 0 \) is the equations that always needs to be false.


Elliptic curve pairings are maps

\[ G_2 \times G_1 \rightarrow G_t  \]

With:
\begin{itemize}
\item \( G_1 \) being an elliptic curve ( \( y^2 = x^3 + b \) ), and \inl{(x, y)} are memebrs of a field \( F_p \).
\item \( G_2 \) being an elliptic curve, where \inl{(x, y)} are members of a field \( F_p^{12} \). We also defined a \( w \) defined by a 12th degree polynomial.
\item \( G_t \) is the result which is also F_p^{12}
\end{itemize}

\

\section{Computing pairings with Miller's Algorithm}
The most optimized way of computing a bilinear and non-degenerate elliptic curve pairing \( e (P, Q) \) where P and Q are two points on an elliptic curve and the result is a scalar, is by first using miller's recursive algorithm to compute a function \( f_{\lambda, Q}(P)\) and then raising it to the large power \( (q^{12} - 1) / r \).

With \( q \) as the prime order of the coordinate field of BLS12 \( F_q \), and \( r \) as the order of the groupt of points on the curve, and two points \( Q, P \), we want to compute \( f_\lambda, Q (P) \) of \( F_q^{12} \) which is simply an evaluation of a certain function \( f_\lambda, Q \) at point \( P \) with \( \lambda \) being the remainder of \( q \div r \).

We compute such rational function with Miller's loop which recursively builds rational functions of the type \( f_i, Q \) with \( i \) between \( 0 \) and \( \lambda \):

\[ f_{ i + j, Q} = f_{ i, Q } \times f_{ j, Q } \times l_{i, j, Q} \div v_{i + j, Q} \]

With \( l_{i, j, Q} \div v_{i + j, Q} \) being the ratio between a line passing through \( iQ, jQ \) and one (vertical) passing through \( (i + j) Q \).

When actually computing the algorithm, there are some optimizations to make:

\begin{itemize}
\item no need to evaluate vertical lines since they will always resolve to 1 after the final exponentiation.
\item we twist to an EC over \( F_{12} \).
\end{itemize}

\subsection{Final Exponentiation}
In order to compute an optimal Ate pairing, when need to raise the output of out miller loop to \( q^{12} - 1 \div r \):

\[ ate (P, Q) = f_{\lambda, Q} (P) ^{(q^{12} - 1) \div r} \]

Since the power \( (q^{12} - 1) \div r \) is very large in bits, using a standard repeated multiplication to raise \( f \) would be inconvenient, thus we use the frobenius operator and the cyclotomic subgroup to split the computation in two parts.

\subsubsection{Forbenius Operator}
The frobenius operator allows us to perform \( f \to f^q \), and is computationally efficient since we work with finite fields.
This way we can factorize the power into:

\[ (q^6 - 1)(q^2 +1) \times ( q^4 - q^2 + 1 \div r ) \]

and use the frobenius operator to obtain:

\[ f_{frobenius}^{(q^6 - 1)(q^2 +1)} \]


\subsubsection{cyclomatic subgroup}
We now have to compute \( f_{frobenius}^{( q^4 - q^2 + 1 \div r )} \), to do so we do:

\[ f_{final} = f_{}\limits{ ^{\lambda_0}} \times (f_{} ^ { \lambda_1})^q \times (f_{} ^ { \lambda_2})^{q^2} \times (f_{} ^{\lambda_3})^{q^3}  \]


\end{document}
